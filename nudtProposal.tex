%%%%% --------------------------------------------------------------------------------
%%
%%                           Document Template of NUDT proposal
%%
%%%%% --------------------------------------------------------------------------------
%% Copyright (C) Hanlin Tan <hanlin_tan@nudt.edu.cn> 
%% This is free software: you can redistribute it and/or modify it
%% under the terms of the GNU General Public License as published by
%% the Free Software Foundation, either version 3 of the License, or
%% (at your option) any later version.
%%%%% --------------------------------------------------------------------------------
%% Last Updated: 2017.01.06
%%%%************************ Document Class Declaration ******************************
%%
%\documentclass{ctexart}

\documentclass{Style/nudtproposal}% thesis template of UCAS
%% Multiple optional arguments:
%% [scheme = plain] % for thesis writing of international students
%% [<singlesided|doublesided|printcopy>] % single-sided, double-sided, or print layout
%% [draftversion] % show draft version information, default is no show
%% [fontset = <adobe|...>] % specify font set, default is automatic detection
%% [standard options for ctex class]
%%%%% --------------------------------------------------------------------------------
%%
\usepackage{amsthm}
\theoremstyle{plain}

\newtheorem{theorem}{定理}
\newtheorem{Proof}{证明}
%%%%************************* Command Define and Settings ****************************
%%

\usepackage{Style/commons}% common settings
%% usage: \usepackage[option1,option2,...,optionN]{commons}
%% Multiple optional arguments:
%% [myhdr] % one available header and footer style, will enable fancyhdr
%% [lscape] % provide landscape layout environment
%% [geometry] % configure page layout by geometry package
%% [list] % enable enhanced list environments, useful for Algorithm and Coding
%% [color] % enable color package to use color, default package is xcolor
%% [background] % enable page background, will auto enable color package
%% [tikz] % enable tikz for complex diagrams, will auto enbale color package
%% [table] % enable a table package for complex tables, default is ctable
%% [math] % enable some extra math packages
\usepackage{Style/custom}% user defined commands
\usepackage[backend=biber, style=Biblio/nudtcaspervector,utf8, sorting=none]{biblatex} % 设定引用格式


\usepackage{makecell}
\usepackage{tikz}

\newcommand*{\circled}[1]{\lower.7ex\hbox{\tikz\draw (0pt, 0pt)%
    circle (.5em) node {\makebox[1em][c]{\small #1}};}}

\usepackage{stackengine}
\newcommand\xrowht[2][0]{\addstackgap[.5\dimexpr#2\relax]{\vphantom{#1}}}

\newcommand{\wuhao}{\fontsize{10.5pt}{\baselineskip}\selectfont}    %五号 

\usepackage{Style/nudtstyle} % 包含作者自定义的格式和命令
\usepackage{bm}

\usepackage{algorithm}
\usepackage{algpseudocode}
% 设置参考文献文件
\addbibresource{Ref/ref.bib}


%%%%% ---------------------------------------------------------------------------------
%%%%% ---------------警告:以上内容请勿随意修改,除非你清楚自己在做什么------------


%%%%% --------------提示:修改本节内容用于设置文档,请仔细阅读---------------------
%% 
%% 编译环境:texlive-2015。
%% 推荐IDE:texstudio(WinXXX 太挫了)。
%% 编译选项:tex编译器选择xelatex, 参考文献编译器选择biber(不能用bibtex)!
%% 以上环境配置经过作者测试,确定可以正常使用。

%%%%% ---------------提示:本参数提供一个参考文献格式BUG的临时解决方案------------
% 是否将参考文献放入表格。这个选项的设置因为参考文献格式有一个暂时无解的BUG。设置为yes之前之前,必须先
% 设置为no编译一次。否则正文中引用数字都是0。
% 最终提交前将此参数改为yes编译一次。注意:设置为yes后编译第二次就会出现正文中引用数字都是0的BUG!
% 如果设置为yes后,还需要修改正文,那么改为no,编译  两次 参考文献才会正常!
\enabletablebib{no}   % 最终提交前将此参数改为yes编译一次! 如需修改正文,改为no后编译两次参考文献才会正常!   

%% 以下参数用于设置文档首页和页眉信息
\proposaltype{doctor}          % 研究生类别:硕士设置为master,博士设置为doctor 
\enabletableofcontents{no}   % 是否生成目录:如果需要目录设置为yes,否则设置为no。我校开题报告默认没有目录
\proposalnumber{\underline{\hbox to 10mm{}}}          % 编号:默认是下划线,如果你知道编号,设为真实编号
\classification{公开}              % 密级:公开,秘密,机密或者绝密
\nudttitle{xxx}{xxx}% 因title一般都很长需要两行,第一参数为第一行内容,第二个参数为第二行内容
\author{xxx}                    % 作者
\authorid{xxx}            % 学号
\advisor{xx}                   % 导师
\advisortitle{xx}         % 职称
\degreetype{xx学}                 % 学位类别
\major{xx}          % 一级学科
\field{xx}                      % 研究方向
\institute{xx}% 学院
\chinesedate{xx~年~xx~月~xx日} % 开题日期
\formdate{二零一八年一月}     % 制表月份 注意:用“〇”可能会出现字体不显示的问题,所以这里改为了“零”

%% 在设置完以上参数后,修改Tex文件下对应文件以完成开题报告。
%%%%% ---------------------------------------------------------------


%%%%% ---------------警告:以下内容请勿随意修改,除非你清楚自己在做什么------------

%%%%******************************** Content *****************************************
%%
\begin{document}
%%
%%%%% --------------------------------------------------------------------------------
%%
%%%%******************************** Frontmatter *************************************
%%
\pagenumbering{roman}% restart page numbers with arabic style
%%% Generate Title
%%
\maketitle

%%%%% --------------------------------------------------------------------------------
%%
%%%%******************************** Mainmatter **************************************
%%

%% 添加正文内容
\pagenumbering{arabic}% restart page numbers with arabic style
\mdfsetup{skipabove=0pt,skipbelow=0pt}
{\kaishu \zihao{5}
%% 包含正文各个章节,请编辑章节文件修改相应的内容
%%%%% --------------------------------------------------------------------------------
%%
%%%%******************************* Main Content *************************************
%%
%%% ++++++++++++++++++++++++++++++++++++++++++++++++++++++++++++++++++++++++++++++++++




\section{学位论文选题的立论依据}


\begin{mdframed}[everyline=true]
	
\subsection{课题来源}
{\kaishu \zihao{5}自拟。}


\subsection{选题依据}
xxx


\subsection{研究意义}
xxxx
\\[15 cm]
\end{mdframed}

%%% ++++++++++++++++++++++++++++++++++++++++++++++++++++++++++++++++++++++++++++++++++
%   \include ?= \input + \clearpage
\clearpage
%%%%% --------------------------------------------------------------------------------
%%
%%%%******************************* Main Content *************************************
%%
%%% ++++++++++++++++++++++++++++++++++++++++++++++++++++++++++++++++++++++++++++++++++




\section{文献综述}
\begin{mdframed}[everyline=true]
在本节中,将介绍与本课题研究最相关的现有工作。
\\[20cm]
\end{mdframed}



%%% ++++++++++++++++++++++++++++++++++++++++++++++++++++++++++++++++++++++++++++++++++
%
\clearpage
%%%%% --------------------------------------------------------------------------------
%%
%%%%******************************* Main Content *************************************
%%
%%% ++++++++++++++++++++++++++++++++++++++++++++++++++++++++++++++++++++++++++++++++++




\section{研究内容}
\begin{mdframed}[everyline=true]

\subsection{研究目标}
xxx
\end{mdframed}

\begin{mdframed}[everyline=true]
\subsection{主要研究内容及拟解决的相关科学问题和技术问题}
(1)主要研究内容
xxx

(2)拟解决的关键科学问题和技术问题

xxx

% % % % % % % % % % % %
\end{mdframed}

\begin{mdframed}[everyline=true]

\subsection{拟采取的研究方法、技术路线、实施方案及可行性分析}
\subsubsection{研究方法}
xxx

\subsection{研究方法技术路线与实施方案}
xxx


\subsubsection{可行性分析}
xxx

\end{mdframed}

\begin{mdframed}[everyline=true]
\subsection{预期创新点}
xxx\\[8cm]

\end{mdframed}


%%% ++++++++++++++++++++++++++++++++++++++++++++++++++++++++++++++++++++++++++++++++++
%
\clearpage
%%%%% --------------------------------------------------------------------------------
%%
%%%%******************************* Main Content *************************************
%%
%%% ++++++++++++++++++++++++++++++++++++++++++++++++++++++++++++++++++++++++++++++++++




\section{研究条件}
\begin{mdframed}[everyline=true]

{\bfseries \kaishu \zihao{5} 开展研究应具备的条件及已具备的条件,可能遇到的困难与问题和解决措施。}

\subsection{实验条件} 
xxx

\subsection{研究基础}
xxx

\subsection{可能遇到的困难和解决措施}
在课题研究中可能遇到的困难有:

xxx\\[15 cm]
\end{mdframed}

%%% ++++++++++++++++++++++++++++++++++++++++++++++++++++++++++++++++++++++++++++++++++
%
\clearpage
%%%%% --------------------------------------------------------------------------------
%%
%%%%******************************* Main Content *************************************
%%
%%% ++++++++++++++++++++++++++++++++++++++++++++++++++++++++++++++++++++++++++++++++++




\section{学位论文工作计划}
{
\noindent
\begin{tabular*}{0.999\textwidth}{| p{0.20\textwidth } <{\centering} | p{0.40\textwidth}  | p{0.311\textwidth}  |}

	\hline 
	\multicolumn{1}{|c|}{起讫日期} & 	\multicolumn{1}{c}{主要完成研究内容} & 	\multicolumn{1}{|c|}{预期成果} \\
	\hline 
	\tabincell{c}{2022年01月 -- \\2022年03月}   &  基础知识学习 &   完成文献搜集与该方向基本知识储备 \\ 
	\hline 
	\tabincell{c}{2022年04月 -- \\2022年06月} &  研究点1 &   完成实验 \\ 
	\hline 
	\tabincell{c}{2022年07月 -- \\2022年08月} &  研究点1 &   发表论文一篇 \\ 
	\hline 
	\tabincell{c}{2022年08月 -- \\2022年09月} &  研究点2 &   完成实验 \\ 
	\hline 
    \tabincell{c}{2022年09月 -- \\2022年10月} &  研究点2 &   发表论文一篇 \\ 
    \hline 
    \tabincell{c}{2022年10月 -- \\2022年11月} &  研究点3 &   完成实验 \\ 
    \hline 
    \tabincell{c}{2022年12月 -- \\2023年01月} &  研究点3 &   发表论文一篇 \\ 
    \hline 
    \tabincell{c}{2023年01月 -- \\2023年02月} &  研究点4 &   完成实验 \\ 
    \hline 
    \tabincell{c}{2023年02月 -- \\2023年03月} &  研究点4 &   发表论文一篇 \\ 
    \hline 
    \tabincell{c}{2023年03月 -- \\2023年04月} &  研究点5 &   完成实验 \\ 
    \hline 
    \tabincell{c}{2023年04月 -- \\2023年05月} &  研究点5 &   发表论文一篇 \\ 
    \hline 
	\tabincell{c}{2023年04月 -- \\2023年06月} &  撰写毕业论文 &  完成毕业论文 \\ 
	\hline 
\end{tabular*} 
\\[1 cm]
{\songti 注:每个子阶段不得超过3个月;预期成果中必须包含成果的形式、数量、质量等可考性指标该计划将作为
论文研究进展检查的依据。}
\indent
}


%%% ++++++++++++++++++++++++++++++++++++++++++++++++++++++++++++++++++++++++++++++++++
}%
\clearpage


%%%%% --------------------------------------------------------------------------------
%%
%%%%******************************* Main Content *************************************
%%
%%% ++++++++++++++++++++++++++++++++++++++++++++++++++++++++++++++++++++++++++++++++++


\section{主要参考文献}
\printbib
%\printbibtabular[title={~}]
%\printbibliography[title={~}]

%%% ++++++++++++++++++++++++++++++++++++++++++++++++++++++++++++++++++++++++++++++++++
%
\clearpage
%%%%% --------------------------------------------------------------------------------
%%
%%%%******************************* Main Content *************************************
%%
%%% ++++++++++++++++++++++++++++++++++++++++++++++++++++++++++++++++++++++++++++++++++




\section{指导教师对开题报告的评语}
\begin{mdframed}[everyline=true]
\indent {\songti{(对1-6项逐项予以评价,并着重对国内/外研究现状的了解情况、研究内容的创新性等方面进行评价,
最终给出是否满足博士/硕士层次学位论文研究要求的综合评价意见)}}

xxx
\\[12 cm]

\begin{flushright}
    \songti{导师签字:}\ \ \ \ \ \ \ \ \ \ \ \ \ \ \ \ \ \ \ \ \ \ \ \ \ \ \ \ \ \ 
    \vspace{10 mm}

    \ \ \ \ \ \ \ \ \ \ \ 年 \ \ \ \ \ \ 月 \ \ \ \ \ \ \ 日
    \vspace{30 mm}
\end{flushright}

\end{mdframed}

%%% ++++++++++++++++++++++++++++++++++++++++++++++++++++++++++++++++++++++++++++++++++
%
\clearpage
%%%%%%% --------------------------------------------------------------------------------
%%
%%%%******************************* Main Content *************************************
%%
%%% ++++++++++++++++++++++++++++++++++++++++++++++++++++++++++++++++++++++++++++++++++


\section{开题报告评议小组意见及评议结果}
\begin{mdframed}[everyline=true]

\begin{enumerate}[label={(\arabic*)},labelsep= 3 pt]
	\item {\songti 选题依据、研究内容、研究方案及技术路线的科学性、可行性及创新性的评价}
	
	\quad\quad  王思为同学的博士开题报告深入分析国内外的理论观点和技术方案,对复杂无监督多视图学习算法和理论研究思路比较清晰,研究方法具有创新性,选题具有重要的应用价值。论文研究内容与工 作量适合博士学位论文的要求,论文研究方法可行。
	
	经评议小组讨论,一致同意王思为同学的博士学位论文开题报告。
	
	\item  {\songti 存在的主要问题和修改建议 }
	\begin{enumerate}[label={\arabic*)},labelsep=3 pt]
		\item 论文不必提出“广义异常事件”的概念,只需说明工作内容为异常事件检测而不是区分 异常事件的具体类别。
		\item 异常事件检测如果不限定应用场景范围难度和工作量较大,建议先设定一个具体的场景 以降低难度,逐步推进。
		\item 光流的导数(加速度)也可以作为特征向量的一个维度。
		\item 如果采用的方法是无监督的可能更具有实用性。
	\end{enumerate}

	\item  {\songti 开题报告评议结果}
	
    $\square$  {\songti 通过} \quad\quad \quad \quad  $\square$ {\songti 不通过,且要求在2个月内重新组织开题}
	\\[20pt]
	
	
	{  \songti
	\quad\quad\quad\quad \quad\quad\quad\quad 	\quad\quad\quad\quad \quad\quad\quad\quad 组长(签名):
	
	\quad\quad\quad\quad \quad\quad\quad\quad 	\quad\quad\quad\quad \quad\quad\quad\quad \quad\quad\quad 年  \quad\quad 月  \quad\quad 日
	\\
    }
\end{enumerate}
\end{mdframed}
\vspace{-3pt}
{
	\noindent
\begin{tabular*}{0.999\textwidth}{| c  | c | c | p{0.5\textwidth}< {\centering} | p{0.157\textwidth}<{\centering}|}
%	\hline 
    \multicolumn{5}{|c|}{	\songti 开题报告评议小组组成}	\\
	\hline
	{\songti 组成} & {\songti 姓名} & {\songti 职称} &  {\songti  所在单位} & {\songti 本人签名}  \\
	\hline 
	{\songti 组长}     &  张老师 & 教授 &  五院XXXX研究所 &  \\ 
	\hline 
	\multirow{4}{8pt}{\songti 成员}  & 李老师  &  教授&  五院XXXX研究所 & \\ 
	\cline{2-5}
	   &  王老师 &  教授&  X院XXXX研究所 & \\ 
	\cline {2-5}
	   &  谭老师 &  教授&  五院XXXX研究所 & \\ 
	\cline {2-5}
	   &  老老师 &  教授&  五院XXXXXX系 & \\ 
	\hline 
	{\songti 秘书}   & 赖老师 & 讲师 &  五院XXXXXX系 & \\ 
	\hline 
\end{tabular*} 
   \indent
}


%%% ++++++++++++++++++++++++++++++++++++++++++++++++++++++++++++++++++++++++++++++++++
%
%5\clearpage

\end{document}
%%%%% --------------------------------------------------------------------------------
